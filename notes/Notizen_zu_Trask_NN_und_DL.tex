\documentclass[a4paper,12pt,headings=small,ngerman,bibliography=totoc]{scrartcl}
\usepackage[ngerman]{babel}
\usepackage[utf8]{inputenc}
\usepackage[headsepline]{scrlayer-scrpage}

\usepackage{amssymb,graphicx}
\clearscrheadings
\pagestyle{scrheadings}
\usepackage[left=2.5cm,right=2.5cm,top=3cm,bottom=2.5cm]{geometry}
\usepackage[onehalfspacing]{setspace}
\usepackage{microtype}
\ofoot{\pagemark}
\footskip1cm
\setkomafont{sectioning}{\normalfont\normalcolor\bfseries}
\usepackage[osf,sc]{mathpazo}
\usepackage{url}
\clubpenalty = 10000
\widowpenalty = 10000 \displaywidowpenalty = 10000

\usepackage[usenames,
	    dvipsnames,
	    svgnames]{xcolor}

\usepackage[babel,german=quotes]{csquotes}
\usepackage{url}
\usepackage{soul}

%%% Für Bildunterschriften %%%
\usepackage{caption}

%%% Für Bilder im Text %%%
\usepackage{wrapfig}

%%% Für Mathe %%%
\usepackage[fleqn]{amsmath}
\usepackage{commath}
\usepackage{amstext}
\allowdisplaybreaks
%%% Für schräge Bruchstriche %%%
\usepackage{nicefrac}
%%% Rechtecke %%%
\usepackage{amsfonts}

%%% Für Textübergreifende Numerierung %%%
\usepackage{enumitem}
\renewcommand{\labelenumi}{\alph{enumi})}

%%% Für Abkürzungen %%%
\usepackage{acronym}

%%%% Überschriften bis subsubsubsection (paragraph) %%%
\setcounter{secnumdepth}{5}

\setkomafont{dictumtext}{\itshape\small}
\setkomafont{dictumauthor}{\small}
\renewcommand*\dictumwidth{\linewidth}
\renewcommand*\dictumauthorformat[1]{--- #1}
\renewcommand*\dictumrule{}

%%% Für Quotes %%%
\usepackage[ngerman]{varioref}
\usepackage{hyperref}
 \hypersetup{%draft, 								% no hyperlinking at all (useful in b/w printouts)
    colorlinks=true, breaklinks=true,
    urlcolor=Black, linkcolor=Black, citecolor=Black,
    linktoc=page, %
    bookmarksnumbered, bookmarksopenlevel=1, bookmarksdepth = section,%
    pdfstartview=FitV,
    }
\setlength{\parindent}{0em}
\usepackage{cleveref}
\crefname{paragraph}{Abschnitt}{Abschnitt}

\usepackage[bibencoding=utf8,sortlocale=de_DE,style=footnote-dw,ibidpage=true,origfieldsformat=brackets,backend=biber]{biblatex}
\addbibresource{bib.bib}
%\renewcommand\bibname{Literaturverzeichnis}

%%% urldate in eckigen Klammern %%%
\DeclareFieldFormat{urldate}{\mkbibbrackets{#1}}
%%% URL: = Verfügbar unter: %%%
\DeclareFieldFormat{url}{{Verfügbar unter:}\space\url{#1}}
%%% Abstand zwischen den Literaturangaben %%%
\setlength{\bibitemsep}{1.3em}
%%% statt und ein & %%%
\renewcommand*{\finalnamedelim}{\space\&\space}
%%% Nachname, Vorname, immer %%%
\DeclareNameAlias{sortname}{last-first}

%%% Kopfzeile %%%
\automark[section]{section}
\renewcommand*{\headfont}{\normalfont}
\setkomafont{pageheadfoot}{}
\renewcommand*{\sectionmarkformat}{} 

%%% Damit das Abbildungsverzeichnis im Inhaltsverzeichnis abgebildet wird %%%
\usepackage{tocbibind}
\renewcommand{\listoffigures}{\begingroup
  \tocchapter
  \tocfile{\listfigurename}{lof}
  \endgroup}
  
%%% Für SI Einheiten %%%
\usepackage{siunitx}
\sisetup{
  locale = DE ,
  detect-all
}

%Für Tabellen
\usepackage{array}
% Für Seitenübergreifende Tabellen
\usepackage{longtable}
% Für Tabellen mit vertikal und horizontal zentriertem Inhalt
\newcolumntype{B}[1]{>{\centering\arraybackslash}m{#1}}

\begin{document}

% \pagenumbering{gobble}
% \pagestyle{empty}
% 
% 
% \begin{center}
% \Large{Hochschule für Technik und Wirtschaft Dresden (FH)}\\
% \end{center}
% 
% \begin{center}
% \Large{Fachbereich Elektrotechnik}
% \end{center}
% \begin{verbatim}
% 
% \end{verbatim}
% \begin{center}
% \textbf{\LARGE{Optoelektronik}}
% \end{center}
% \begin{verbatim}
% 
% \end{verbatim}
% \begin{center}
% \textbf{im Studiengang\\
%  Mechatroniksysteme/Fahrzeugmechatronik}
% \end{center}
% \begin{verbatim}
% 
% \end{verbatim}
% 
% \begin{flushleft}
% \begin{tabular}{lll}
% & & \\
% & & \\
% & & \\
% & & \\
% & & \\
% & & \\
% \textbf{erstellt von:} & Johannes Leyrer \flq{}Johannes.Leyrer+OE@gmail.com\frq{}\\
% & & \\
% & & \\
% \textbf{erstellt im:} & Wintersemester 2017/2018\\
% & & \\
% & & \\
% \end{tabular}
% \end{flushleft}
% 
% \vspace{\fill}
% \begin{description}
% \item[Disclaimer:] Diese Mitschrift ist nicht offiziell. Insbesondere erhebe ich keinen Anspruch auf Vollständigkeit oder Korrektheit. Ich bin jederzeit froh um Hinweise zu Fehlern oder Unklarheiten.
% \end{description}
% 
% \newpage
% %%% Inhaltsverzeichnis einfügen %%%
% \tableofcontents
% \cleardoublepage
% 
% \newpage
% \pagestyle{scrheadings}
% \pagenumbering{arabic}
% \ohead{Johannes Leyrer, WS 17/18}
% \ihead{\rightmark}

\section{Einleitung}

Das ist ein Test. :) \cite{Trask}


\newpage

\printbibliography[title={Literaturverzeichnis}]

\end{document}
